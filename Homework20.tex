\documentclass{article}     
\usepackage[utf8]{inputenc} 
\usepackage{amsfonts}
\usepackage[left=3cm,right=3cm,
    top=3cm,bottom=4cm,bindingoffset=0cm]{geometry}
\sloppy
\usepackage[T2A]{fontenc}
\usepackage{amsmath}
\title{Домашнее задание по дискретной математике}  
\author{Родигина Анастасия, 167 группа}     
\date{26 февраля  2017}   

\usepackage{graphicx}
\graphicspath{{pictures/}}
\DeclareGraphicsExtensions{.pdf,.png,.jpg}

\newcommand{\ip}[2]{(#1, #2)}
                             
\begin{document}            

\maketitle  
 \noindent \textbf{Задача 1}
\begin{center} 
\textit{Верно ли, что множество наборов подряд идущих цифр длины 5, входящих в десятичную запись числа $\pi$ а) перечислимо; б) разрешимо? Можно считать, что любая цифра числа $\pi$ вычислима.}
\end{center}
Для начала докажем пункт (б). Заметим, что число различных наборов цифр длины 5 конечно (Всего их $10^5$). Воспользуемся теоремой, что любое конечное множество разрешимо. 
Таким образом, множество наборов подряд идущих цифр длины 5, входящих в десятичную запись числа $\pi$ разрешимо.\\
(а) Так как мы доказали разрешимость этого множества, воспользуемся теоремой, что любое разрешимое множество является перечислимым, что и требовалось доказать.
\newline
\newline
\textbf{Задача 2}
\begin{center}
\textit{Пусть множество $X$ натуральных чисел перечислимо. Перичислимо ли множество $Y \subseteq X$ тех чисел из $X$, у которых сумма цифр равна 10.}
\end{center}
Если множество $X$ натуральных чисел перечислимо, то существует алгоритм, который перечисляет элементы этого множества. Возьмем и модифицируем этот  алгоритм, будем перечислять элементы множества $X$, но каждый раз будем осуществлять проверку на сумму цифр. Если сумма цифр будет равна 10, то будем выводить это число. Таким образом, мы нашли алгоритм, который сможет перечислить элементы множества $Y$, а, значит, по опрделению перечислимого множества, $Y$ - перечислимо. 
 \newline
 \newline
\textbf{Задача 3}
\begin{center}
\textit{Докажите, что если $A,~B$ - перечислимые множества, то и множество $A\times B$ перечислимо.} 
\end{center}
Для каждого из множеств будет существовать алгоритм, который в каком-то порядке будет перечислять элементы множества. Каждому элементу присвоим нумерацию, которая показывает, на какой итерации алгоритма мы получаем это число. $a_i$,
$b_i$ - элементы, получаемые из $A$ и $B$ соответственно на i-м шаге. Будем последовательно рассматривать элементы множеств. При каждом $k \in \mathbb{N}$ $\forall i \in \{0..k\}$ Напечатаем элементы ($a_{k-i},~b_i$). Для этого потребуется конечное число шагов. Далее переходим к числу $k+1$. Теперь рассмотрим произвольную пару $(a_n,~b_m),~a \in A,~ b\in B$. Она будет напечатана при $k=m+n$ и $i=m \leq k$
\newline
\newline
 \noindent \textbf{Задача 4}
\begin{center}
\textit{Всюду определенная функция $f: \mathbb{N} \rightarrow\mathbb{N}$ строго возрастает и множество ее значений содержит все натуральные числа за исключение конечного множества. докажите, что $f$ вычислима.}
\end{center}
 Обозначим множество значений этой функции как $A$. Так как любое конечное множество является перечислимым, получаем, что $\overline{A}$ является перечислимым. Кроме того, само $A$ является перечислимым. По теореме Поста, получаем, что $A$ разрешимо. Теперь воспользуемся фактом, что функция строго возрастает. Переберем натуральные числа с 0, проверяя, принадлежит ли это число множеству $A$. Каждое число будет соответствовать значению функции на i, где i - количество предшествующих единиц в характеристической функции, таким образом, функция вычислима.
\newline \newline
 \noindent \textbf{Задача 5}
\begin{center}
\textit{Существуют ли такие множества $X,~Y \subseteq \mathbb{N}$, что $X$ рарешимо, $X\cup Y$ разрешимо, а $Y$ не разрешимо?} 
\end{center}
Приведем пример такого множества. Возьмем, множество $X$, совпадающее с $\mathbb{N}$ и любое неразрешимое множество $Y$ (гарантировано, что такое множество найдется, доказывалось на лекции). Получаем, что $X\cup Y=X$ является разрешимым.
\newline \newline
 \noindent \textbf{Задача 6}
\begin{center}
\textit{Пусть $S$ - разрешимое множество натуральных чисел. Множество $D$ состоит из всех простых делителей множества $S$. Верно ли, что $D$ перечислимо?}
\end{center}
Так как любое разрешимое множество является перечислимым, будем перечислять элементы множества $S$, выписывая все простые делители каждого элемента (а это можно сделать за конечное количество шагов). Если $0 \in S$, то просто записываем в  $D$ простые числа, множество которых является перечислимым. Таким образом, мы можем перечислить все элементы множества $D$, значит, оно перечислимо. 
\newline
\newline \newline
 \noindent \textbf{Задача 7}
\begin{center}
\textit{Пусть $f$ -  вычислимая биекция между $\mathbb{N}$ и $\mathbb{N}$. Докажите, что обратная биекция $f^{-1}$ также вычислима.}
\end{center}
Будем перечислять натуральные числа в порядке возрастания. Так как функция $f$ вычислима, сопоставим каждому натуральному числу значение функции от этого натурального числа. Таким образом мы перечислим все значения $f^{-1}$ и все их прообразы.
Запишем алгоритм для вычисления $f^{-1}(x)$. Будем перебирать значения из $\mathbb{N}$ пока значение $f$ от этого элемента не совпадет с $x$, это гарантированно произойдет, тк $f$ - биекция.
Получаем, что обратная биекция $f^{-1}$ вычислима.
\newline
\newline 
\end{document}
