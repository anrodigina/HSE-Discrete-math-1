\documentclass{article}     
\usepackage[utf8]{inputenc} 
\usepackage{amsfonts}
\usepackage[left=4cm,right=4cm,
    top=3cm,bottom=4cm,bindingoffset=0cm]{geometry}
\sloppy
\usepackage[T2A]{fontenc}
\usepackage{amsmath}
\title{Домашнее задание по дискретной математике}  
\author{Родигина Анастасия, 167 группа}     
\date{9 марта  2017}   

\usepackage{graphicx}
\graphicspath{{pictures/}}
\DeclareGraphicsExtensions{.pdf,.png,.jpg}

\newcommand{\ip}[2]{(#1, #2)}
                             
\begin{document}            

\maketitle  
 \noindent \textbf{Задача 1}
\begin{center} 
\textit{Пусть $U(p,x)$ - главная универсальная вычислимая функция. Докажите, что найдется бесконечно много таких $p$, что $U(p,x)=2017$ для какого-то $x$.}
\end{center}
Возьмем такую функцию $f(x)=2017$, которая определена на каком-то одном натуральном значении $x$. Заметим, что множество натуральных чисел является бесконечным множеством, а,значит, и множество таких функций бесконечно. Это дает основания утверждать, что найдется беконечно много таких $p$, что $U(p,x)=2017$ для какого-то $x$.
\newline
\newline
\textbf{Задача 2}
\begin{center}
\textit{Пусть $U(p,x)$ - главная универсальная вычислимая функция. Докажите, что найдется такое $n$, что $U(n,x)=nx$ для всех $x$.}
\end{center}
 Возьмем, что $V(n,x)=nx$. Заметим, что задача 2 является частным случаем третьей задачи, которая как раз показывает существование такого $n$, что $U(n,x)=nx$ для всех $x$.
 \newline
 \newline
\textbf{Задача 3}
\begin{center}
\textit{Пусть $U(p,x)$ - главная универсальная вычислимая функция, а $V(n,x)$ - вычислимая функция двух аргументов. Докажите, что найдется такое $p$, что $U(p,x)=V(p,x)$ для всех $x$.} 
\end{center}
По определению главной универсальной функции найдется такая функция $s(x)$ $\forall n,x$, что:
$$V(n,x)=U(s(n),x)$$
Воспользуемся теоремой о неподвижной точке:
$$\exists t : U(s(t),x)=U(t,x)$$
Таким образом получаем:
$$V(t,x)=U(s(t),x)=U(t,x)$$
\newline
\newline
 \noindent \textbf{Задача 4}
\begin{center}
\textit{Существует ли такая главная универсальная функция $U(p,x)$, в которой множество программ $I$, вычисляющих определенные в 0 функции, совпадает с множеством четных чисел?}
\end{center}
 Если множество программ совпадает с множеством четных чисел, то достаточно очевидно, что оно будет являться разрешимым, так как проверить число на четность не так уж и сложно. Воспользуемся теоремой Успенского-Райса. За нетривиальное свойство возьмем определенность функции в 0, тогда получаем, что множество номеров p таких программ в главной нумерации $U(p,x)$ является неразрешимым множеством. Отсюда получаем противоречие. Таким образом, такой функции не существует.
\newline \newline
 \noindent \textbf{Задача 5}
\begin{center}
\textit{Тот же вопрос про неглавную нумерацию.} 
\end{center}
Любая главная универсальная функция является универсальной функцией, значит, мы можем воспользоваться задачей 4, любая главная нумерация является и неглавной нумерацией, а пример существования мы привели в про
\newline \newline
 \noindent \textbf{Задача 6}
\begin{center}
\textit{Пусть $U(p,x)$ - главная универсальная вычислимая функция. Обозначим через $K \subset\mathbb{N}^2$ множество таких пар $(k,n)$, что функция $U_k(x)=U(k,x)$ является продолжением функции $U_n(x)=U(n,x)$ (т.е. $U(k,x)=U(n,x)$ если $U(n,x)$ определена). Докажите, что множество $K$ неразрешимо.}
\end{center}
Возьмем такие функции $f(1)=1,~\forall i\neq1 ~~f(i) -$ не определена. Тогда $\exists n : U_n(x)=f(x)$.
Пусть
$K'=\{k~|~U_k-$ продолжение $U_n\}$. Данное свойство нетривиально, т.к. существуют как продолжения $U_n$, так и не продолжения. Поэтому мы можем воспользоваться теоремой Успенского-Райса. Отсюда получаем, что $K'$ - неразрешимо.
Теперь предположим, что $K$ разрешимо. Тогда запустим алгоритм вычисления его характеристической функции на $n$ и произв. $x$. Тогда можно будет сказать является ли $U_n(x)$ продолжением $U(n,x)$. Так как мы зафиксировали $n$ (см. второе предл.), то на вход подаем $x$ - из этого получаем алгоритм, вычисляющий характеристическую функцию на $n,x$. Отсюда получаем, что $K'$ - разрешимо. Получаем противоречие
\newline \newline
 
\end{document}
