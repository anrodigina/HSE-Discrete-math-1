\documentclass{article}     
\usepackage[utf8]{inputenc} 
\usepackage{amsfonts}
\usepackage[left=4cm,right=4cm,
    top=3cm,bottom=4cm,bindingoffset=0cm]{geometry}
\sloppy
\usepackage[T2A]{fontenc}
\usepackage{amsmath}
\title{Домашнее задание по дискретной математике}  
\author{Родигина Анастасия, 167 группа}     
\date{1 марта 2017}

\usepackage{graphicx}
\graphicspath{{pictures/}}
\DeclareGraphicsExtensions{.pdf,.png,.jpg}

\newcommand{\ip}[2]{(#1, #2)}
                             
\begin{document}            

\maketitle  
 \noindent \textbf{Задача 1}
\begin{center} 
\textit{Докажите, что для любой универсальной вычислимой функции $U$ множество $\{U(p,p)~:~p \in \mathbb{N}\}$ совпадает с $\mathbb{N}$}
\end{center}
Достаточно показать, что значения $U(p,p)$ пробегают все натуральные числа. Возьмем произвольное число $n \in \mathbb{N}$ и покажем, что оно может быть достигнуто. Так как все вычислимые функции одного аргумента встречаются среди $U_n$, возьмем такую функцию, которая вне зависимости от входа принимает значение n. Возьмем в качестве p номер этой функции. Тогда $\forall q ~~~~U(p,q)=n$. Получаем, что все  $n \in \mathbb{N}$ достижимы, т.е. множество $\{U(p,p)~:~p \in \mathbb{N}\}$ совпадает с $\mathbb{N}$.
\newline
\newline
\textbf{Задача 2}
\begin{center}
\textit{Верно ли, что для любой универсальной вычислимой функции $U$ множество\\
а) $\{p~|~U(p,p^2)$ определено$ \}$\\
б) $\{p~|~U(p^2,p)$ определено$ \}$}
не разрешимо?
\end{center}
а) Возьмем, что это множество разрешимо. Тогда функция: $y(x)=U(x,x^2)$ - вычислимая функция, и соответственно $f(x)=y(x)+1=U(x,x^2)+1$ - вычислимая функция. Заметим, что функция:
\begin{equation*}
    g(x)=
    \begin{cases}
    0 & ([\sqrt{x}])^2 \neq x\\
    f(\sqrt{x}) & ([\sqrt{x}])^2 = x
    \end{cases}
\end{equation*}
 также будет вычислимой, а следовательно $\exists n :~~ U(n,x)=g(x)~~ \forall x$. Тогда при $x=n^2 ~~: ~~g(n^2) = U(n, n^2) = U (n, n^2) + 1$ Противоречие.\\
 б) Достаточно привести такой пример, что множество разрешимо. Пусть $U(n,x)$ - универсальная функция:
 \begin{equation*}
     V(n,x)=
     \begin{cases}
     0 & ([\sqrt{x}])^2 \neq x\\
     U(f(n),x) & ([\sqrt{x}])^2 = x
     \end{cases}
 \end{equation*}
 где $f(x)$ сопоставляет каждому числу, не являющимся квадратом, его порядковый номер в упорядоченном множестве чисел - неквадратов. Важно заметить, что $V(n,x)$ является универсальной (за счет возможности подобрать нужный первый аргумент), $U(p^2,p)=0$ - определена.  
 \newline
 \newline
\textbf{Задача 3}
\begin{center}
\textit{Докажите, что во всяком бесконечном разрешимом множестве натуральных чисел есть перечислимое неразрешимое подмножество.} 
\end{center}
Элементы бесконечного разрещимого множества натуральных чисел можно перечислить в порядке возрастания: $a_1, a_2,..a_m ~|~ \forall i: a_i<a_{i+1}$ тогда функция $f(i)=a_i$ - вычислима. Существование перечислимых неразрешимых подмножеств множества натуральных чисел доказывалось на лекции. Возьмем такое подмножество $S$. Заметим, что множество $f(S) \subset A$ является перечислимым за счет существования функции $f$. Теперь покажем то, что оно действительно будет являться неразрешимым. Предположим обратное. Тогда будет существовать какая-то характеристическая функция, проверяющая, находится ли $f(n)$ в $ f(S)$, тогда можно было бы проверить находится ли $n$ в $S$ (из монотонности), что противоречит неразрешимости $S$.
\newline
\newline
\noindent \textbf{Задача 4}
\begin{center}
\textit{Докажите, что бесконечное подмножество $\mathbb{N}$ разрешимо тогда и только тогда, когда оно является областью значений всюду определенной возрастающей вычислимой функции из $\mathbb{N}$ в $\mathbb{N}$.}
\end{center}
(=>) Возьмем произвольное бесконечное разрешимое подмножество натуральных чисел $S$. Перечислим элементы этого множества, проверяя элементы множества натуральных чисел от 0 в порядке возрастания. Таким образом, мы выпишем возрастающую последовательность $\{a_i\}$. Возьмем функцию $f(i)=a_i$, очевидно, что она является вычислимой, а $S$ является множеством значений этой функции.\\
(<=) Запишем значения всюду определенной возрастающей вычислимой функции из $\mathbb{N}$ в $\mathbb{N}$, проходя элементы множества натуральных чисел от 0 в порядке возрастания. Значения, которые мы выписали будут находиться в порядке возрастания. По теореме из учебника (что если существует алгоритм перечисления элементов множества S в возрастающем порядке, то это множество разрешимо) получаем, что это подмножество разрешимо.
\newline \newline
 \noindent \textbf{Задача 5}
\begin{center}
\textit{Докажите, что любое бесконечное перечислимое множество содержит бесконечное разрешимое подмножество.} 
\end{center}
Запишем первый элемент множества при перечислении элементов множества. Затем запишем следующий элемент, который будет больше, чем предыдущий записанный, будем повторять эту операцию по мере перечисления. Получаем перечислимое множество, элементы которого находятся в порядке возрастания. По теореме из учебника (что если существует алгоритм перечисления элементов множества S в возрастающем порядке, то это множество разрешимо) получаем, что это множество разрешимо.
\newline \newline
 \noindent \textbf{Задача 6}
\begin{center}
\textit{Докажите, что перечислимо множество программ, которые останавливаются хотя бы на одном входе. Более формально: пусть $U$ - универсальная вычислимая функция, а $S$ - множество тех $p$, для которых функция $U(p,x)$ определена хотябы при одном $x$. Тогда $S$ перечислимо.}
\end{center}
Обозначим три параметра: $p$ - номер алгоритма вычисляющий какую-либо функцию, $x$ - входные параметры, $t$ - количество шагов, за которое программа останавливается. Будем перебирать эти три параметра, если программа будет останавливать свою работу за $t$ шагов, то будем записывать ее номер. Из того, что множество пар натуральных чисел перечислимо, следует и то, что тройки натуральных чисел перечислимы. А следовательно есть алгоритм, который будет перечислять такие программы.
\newline
 \newline
\end{document}
