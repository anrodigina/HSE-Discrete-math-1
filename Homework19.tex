\documentclass{article}     
\usepackage[utf8]{inputenc} 
\usepackage{amsfonts}
\usepackage[left=4cm,right=4cm,
    top=3cm,bottom=4cm,bindingoffset=0cm]{geometry}
\sloppy
\usepackage[T2A]{fontenc}
\usepackage{amsmath}
\title{Домашнее задание по дискретной математике}  
\author{Родигина Анастасия, 167 группа}     
\date{15 февраля  2017}   

\usepackage{graphicx}
\graphicspath{{pictures/}}
\DeclareGraphicsExtensions{.pdf,.png,.jpg}

\newcommand{\ip}[2]{(#1, #2)}
                             
\begin{document}            

\maketitle  
 \noindent \textbf{Задача 1}
\begin{center} 
\textit{Постройте схему полиномиального размера для функции $f : \{0, 1\}^{(^n_2)} \rightarrow {0, 1}$, равной единице, тогда и только тогда, когда в данном на вход графе есть изолированные вершины.}
\end{center}
Зададим граф верхним треугольником матрицы смежности таким образом: $a_{ij} = 1$, если есть ребро между вершинами $i$ и $j$, а иначе $a_{ij} = 0$. Запишем условие изолированности вершины: $\forall j < i: x_{ji}=0;~\forall j>i:x_{ij}=0$
Запишем проверку на изолированность вершины k в виде логического высказывания:
$$\overline{\big(\bigvee ^{k}_{i=0}x_{ik}\big)\vee \big(\bigvee ^{n}_{i=k+1}x_{ki}\big)}$$
Проверим изолированность остальных вершин:
$$\bigvee ^{n}_{k=0}\neg\big(\big(\bigvee ^{k}_{i=0}x_{ik}\big)\vee \big(\bigvee ^{n}_{i=k+1}x_{ki}\big)\big)$$
Оценим размер схемы за $O(n^2)$, так как количество операций на каждой вершине будет меняться как арфметическая прогрессия от ) до n.
\newline
\newline
\textbf{Задача 2}
\begin{center}
\textit{Треугольником в графе называется тройка вершин, попарно соединенных между собой. Постройте схему полиномиального размера для функции $f : \{0, 1\}^{(^n_2)} \rightarrow {0, 1}$, равной единице, тогда и только тогда, когда в данном на вход графе нет треугольников.}
\end{center}
Зададим граф верхним треугольником матрицы смежности таким образом: $a_{ij} = 1$, если есть ребро между вершинами $i$ и $j$, а иначе $a_{ij} = 0$. Запишем условие нахождения таких треугольников: $\exists i<j<k: x_{ij} = 1,~ x_{ik}=1,~x_{jk}=1$
Запишем схему в виде логического высказывания:
$$\neg \big(\bigvee^n_{i=0~~~}\bigvee^n_{~j=i+1} \bigvee^n_{~k=j+1}\big(x_{ij}\wedge x_{ik}\wedge x_{jk}\big) \big)$$
Оценим размер схемы за $O(n^3)$, тк перебираются три вершины
\newline 
\newline
 \noindent \textbf{Задача 3}
\begin{center}
\textit{Постройте схему полиномиального размера для функции $f : \{0, 1\}^{(^n_2)} \rightarrow {0, 1}$, равной единице, тогда и только тогда, когда данный на вход граф связен и содержит эйлеров цикл.} 
\end{center}
Зададим граф верхним треугольником матрицы смежности таким образом: $a_{ij} = 1$, если есть ребро между вершинами $i$ и $j$, а иначе $a_{ij} = 0$. Граф содержит Эйлеров цикл, если он связен и все степени вершин четны. Так как проверка на связность проводилась на лекции (с помощью возведения матрицы смежности в степень), возьмем, что исходный граф связен. Для каждой вершины проверим четность вершин, используем сложение по модулю 2
$$\bigwedge ^{n}_{k=0}\neg\big(\bigoplus ^{k}_{i=0}x_{ik}\oplus \bigoplus ^{n}_{i=k+1}x_{ki}\big)$$
Оценим сложность схемы за $O(n^2)$
\newline
\newline
  \noindent \textbf{Задача 4}
\begin{center}
\textit{Докажите, что любую монотонную функцию от n переменных можно вычислить схемой размера $O(n2^n)$, используя только дизъюнкцию и конъюнкцию.}
\end{center}
Воспользуемся утверждением, доказанным на семинаре, что любая монотонная функция представима в ДНФ без использования отрицания. Максимальное количество элементов в каждой конъюнкции - $n$. Максимальное количество дизъюнкций конъюнктов - $2^n$. Отсюда получаем, что количество операций будет составлять $O(n2^n)$
\newline \newline
 \noindent \textbf{Задача 5}
\begin{center}
\textit{Докажите, что существует функция от n переменных $(n > 2)$, не вычисляющаяся в базисе $\{\oplus, \cdot, 1\}$ схемой размера $n^{100}$.} 
\end{center}
Не задача, а гроб какой-то:(
\newline \newline
 \noindent \textbf{Задача 6}
\begin{center}
\textit{Докажите, что в базисе $\{\oplus, \cdot, 1\}$ любая функция от n переменных вычисляется схемой размера не более $2^{n+1}$.}
\end{center}
Используем факт, что любую функцию можно представить в виде полинома Жегалкина в базисе $\{\oplus, \cdot, 1\}$:
$$f(x_1, x_2,..,x_n) = a \oplus a_1 \cdot x_1 \oplus .. \oplus a_n \cdot x_n \oplus a_{1,2} \cdot x_1 \cdot x_2... a_i \in \{0, 1\}$$
В этом полиноме будет $2^n$ слагаемых, их $\oplus$ вычисляется за $2^n - 1$. Каждое слагаемое вычисляется реккурентно. Вычислив произведение $k$ элементов, мы выбираем следующий $C^n_{n-k}$ способами, при этом мы делаем ровно 1 операцию, вычисляя произведение. $\sum^n_{i=0}C^n_{n-k}=2^n$ Получаем:
$$2^n+2^n-1=2^{n+1} -1 < 2^{n+1} $$
\newline
\newline
 \noindent \textbf{Задача 7}
\begin{center}
\textit{Булева функция $f : \{0, 1\}^n \to \{0, 1\}$ называется линейной, если она представляется в виде $$f(x_1,..,x_n)=a_0 \oplus(a_1 \wedge x_1)\oplus..\oplus(a_n \wedge x_n)$$
для некоторого набора $(a_1,.., a_n) \in \{0, 1\}^n$ булевых коэффициентов.
Докажите, что схема, использующая только линейные функции, вычисляет линейную функцию}
\end{center}
Докажем это по индукции (по количеству функций в схеме).
База индукции: k=1 - достаточно очевидно, что если схема содержит только одну лин. функцию, то утверждение верно
Индукционный шаг: $k \rightarrow k+1$
Пусть все предыдущие $k$ функций линейны:
$$\forall j\in\{1,..k\}: f_k (x_1,..,x_n)=a_{i0} \oplus(a_{i1} \wedge x_{i})\oplus..\oplus(a_{in} \wedge x_n)$$
Теперь выразим $k+1$ функцию через $k$ предыдущих: 
$$f_{k+1}=a_{k+1,0}\oplus (a_{k+1,1}\wedge a_{1,0}\oplus a_{1,1}\wedge x_1\oplus..\oplus a_{1,n}\wedge x_n) \oplus .. \oplus (a_{k+1,k}\wedge a_{k,0}\oplus a_{k,1}\wedge x_1\oplus..\oplus a_{k,n}\wedge x_n) = $$
$$=\bigoplus_{i=0}^k( a_{k+1,i} \cdot \bigoplus_{j=0}^n a_{i,j}) \oplus \bigoplus_{j=0}^k x_j \bigoplus_{i=0}^n a_{j,i} = b_0 \oplus (b_1 \wedge x_1)\oplus..\oplus(b_n \wedge x_n)$$
Таким образом, утверждение доказано
\newline
\newline 
\noindent \textbf{Задача 8}
\begin{center}
\textit{Докажите, если $f(x_1,...,x_n)$   нелинейная функция, то конъюнкция $x_1\wedge x_2$ вычисляется схемой в базисе $\{0, 1, \neg, f \}$.}
\end{center}
Запишем функцию с помощью полинома Жегалкина. Из нелинейности функции следует, что в полиноме существует хотя бы одно слагаемое вида: $x_i\cdot x_j$:
$$f(x_1,...,x_n) = x_i\cdot x_j\cdot f_1(x_1..x_{i-1},x_{i+1}..x_{j-1}..x_{j+1}..x_n)\oplus $$ 
$$\oplus x_i \cdot f_2(x_1..x_{i-1},x_{i+1}..x_{j-1}..x_{j+1}..x_n)\oplus x_j \cdot f_3(x_1..x_{i-1},x_{i+1}..x_{j-1}..x_{j+1}..x_n)\oplus $$
 $$\oplus f_4(x_1..x_{i-1},x_{i+1}..x_{j-1}..x_{j+1}..x_n) = $$
$$= x_i \cdot x_j \cdot f_1(x_1..x_{i-1},x_{i+1}..x_{j-1}..x_{j+1}..x_n) \oplus Ax_i \oplus Bx_j \oplus C$$
Мы взяли, что  $f_1(x_1..x_{i-1},x_{i+1}..x_{j-1}..x_{j+1}..x_n) \neq 0$ (из $\exists x_i \cdot x_j$ среди слагаемых):
$$f(x_1,...,x_n) = x_i \cdot x_j \oplus Ax_i \oplus Bx_j \oplus C$$ Рассмотрим возможные случаи\\
1) $A, B,C$ равны 0. Получаем конъюнкцию\\
2) $A, B,C$ не равны 0. Тогда построим функцию 
$$f (x_i,x_j) = q(x_1 \oplus Bx_j \oplus A) = (x_i \oplus B) \cdot (x_j \oplus A) \oplus A(x_i \oplus B) \oplus B(x_j \oplus A) \oplus C = $$
$$ = x_i x_j \oplus Ax_i \oplus Bx_j \oplus Ax_i \oplus AB \oplus Bx_j \oplus AB \oplus C = x_i x_j \oplus AB\oplus C$$ 
В зависимости от значения $AB\oplus C$ мы получаем либо конъюнкцию, либо ее отрицание (базис нам позволяет снова применить отрицание, чтобы получить конъюнкцию) 
\newline
 \newline
\end{document}
