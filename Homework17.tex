\documentclass{article}     
\usepackage[utf8]{inputenc} 
\usepackage{amsfonts}
\usepackage[left=4cm,right=4cm,
    top=3cm,bottom=4cm,bindingoffset=0cm]{geometry}
\sloppy
\usepackage[T2A]{fontenc}
\usepackage{amsmath}
\title{Домашнее задание по дискретной математике}  
\author{Родигина Анастасия, 167 группа}     
\date{31 января 2017}   

\usepackage{graphicx}
\graphicspath{{pictures/}}
\DeclareGraphicsExtensions{.pdf,.png,.jpg}

\newcommand{\ip}[2]{(#1, #2)}
                             
\begin{document}            

\maketitle  
 \noindent \textbf{Задача 1}
\begin{center} 
\textit{Рассмотрим бесконечные последовательности из 0, 1 и 2, в которых ни одна цифра не встречается два раза подряд. Какова мощность множества таких последовательностей?}
\end{center}
Закодируем такие последовательности следующим образом. Первый элемент новой последовательности будет равен первому числу последовательности. Заметим, что вариантов выбрать каждый следующий элемент ровно два, если выбираем меньший вариант из двух возможных, то в новой последовательности пишем ноль, иначе 1. Таким образом мы однозначно задаем последовательность  в которой ни одна цифра не встречается два раза подряд. Обозначим наше множество за $A$. Оно равномощно декартову произведению конечного множества и множества бесконечных двоичных последовательностей. Заметим, что:
$$|\mathbb{R}| \leq |A| \leq |\mathbb{R} \times \mathbb{R}| = |\mathbb{R}|$$
Получаем, что наше множество континуально.
\newline
\newline
\textbf{Задача 2}
\begin{center}
\textit{Докажите, что множество отношений эквивалентности на множестве натуральных чисел имеет мощность континуум.}
\end{center}
Используем утверждение, доказанное на лекции, что множество всех бинарных отношений  $\mathbb{N}\mapsto \mathbb{N}$ является континуальным. Так как множество отношений эквивалентности является подмножеством всех бинарных отношений, мощность нашего множества не более чем континуально. Рассмотрим такие отношения эквивалентности, которые разбивают $\mathbb{N}$ на два класса: элементы, состоящие в этом классе эквивалентности и остальные элементы. Такое множество будет равномощно множеству бесконечных двоичных последовательностей (для i-го числа ставим 0 на i-м, если оно не находится в классе эквивалентности, и 1 иначе) - т.е. оно континуально. Получаем, что множество всех отношений эквивалентности не менее, чем континуально. Таким образом, множество отношений эквивалентности на множестве натуральных чисел имеет мощность континуум.
 \newline
 \newline
\textbf{Задача 3}
\begin{center}
\textit{Найдите мощность множества отношений эквивалентности, определенных на множестве действительных чисел.} 
\end{center}
Воспользуемся таким утверждением, доказанным на семинаре, что мощность всех бинарных отношений $\mathbb{R}\mapsto \mathbb{R}$ равномощно $2^{\mathbb{R}}$. Так как множество отношений эквивалентности является подмножеством всех бинарных отношений, мощность нашего множества не более чем $2^{\mathbb{R}}$. Рассмотрим такие отношения эквивалентности, которые разбивают $\mathbb{R}$ на два класса: элементы, состоящие в этом классе эквивалентности и остальные элементы. Такое множество будет равномощно множеству всех подмножеств $\mathbb{R}$, т. е. будет иметь мощность $2^{\mathbb{R}}$. Получаем, что наше множество имеет мощность не менее $2^{\mathbb{R}}$. Таким образом, множество отношений эквивалентности на множестве натуральных чисел имеет мощность $2^{\mathbb{R}}$.
\newline
\newline
 \noindent \textbf{Задача 4}
\begin{center}
\textit{Запишите ДНФ, которая равна булевой функции
$$(x_1 \vee x_2)\wedge(\bar{x}_1 \vee x_3)\wedge(\bar{x}_2 \vee x_4)\wedge(\bar{x}_3 \vee x_5)\wedge...\wedge(\bar{x}_7 \vee x_9)$$}
\end{center}
Воспользуемся свойством коммутативности конъюнкций и последовательно (по частям) раскроем скобки:
Упростим выражение:
$$(\bar{x}_1 \vee x_3)\wedge(\bar{x}_3 \vee x_5)\wedge(\bar{x}_5\vee x_7)\wedge(\bar{x}_7\vee x_9)=$$
$$=(\bar{x}_1\bar{x}_3\vee x_3\bar{x}_3 \vee \bar{x}_1x_5 \vee x_3 x_5)(\bar{x}_5 \bar{x}_7 \vee \bar{x}_5 x_9 \vee x_7 \bar{x}_7 \vee x_7 x_9)=$$
$$=(\bar{x}_1\bar{x}_3\vee  \bar{x}_1x_5 \vee x_3 x_5)(\bar{x}_5 \bar{x}_7 \vee \bar{x}_5 x_9  \vee x_7 x_9)=$$
$$=\bar{x}_1 \bar{x}_3 \bar{x}_5 \bar{x}_7 \vee \bar{x}_1 \bar{x}_3 \bar{x}_5 x_9 \vee\bar{x}_1 \bar{x}_3x_7 x_9 \vee \bar{x}_1 x_5 x_7 x_9 \vee x_3 x_5 x_7 x_9$$
Проведем конъюнкцию между полученным результатом и $(x_1 \vee x_2)$:
$$x_1 x_3 x_5 x_7 x_9 \vee \bar{x}_1 x_2 \bar{x}_3 \bar{x}_5 \bar{x}_7 \vee \bar{x}_1 x_2 \bar{x}_3 \bar{x}_5 x_9 \vee\bar{x}_1 x_2 \bar{x}_3x_7 x_9 \vee \bar{x}_1 x_2 x_5 x_7 x_9 \vee x_2 x_3 x_5 x_7 x_9$$
Теперь упростим выражение: 
$$(\bar{x}_2 \vee x_4)(\bar{x}_4 \vee x_6)(\bar{x}_6 \vee x_8)=(\bar{x}_2 \bar{x}_4 \vee \bar{x}_2 x_6 \vee x_4 x_6)(\bar{x}_6\vee x_8)=$$
$$=\bar{x}_2 \bar{x}_4 \bar{x}_6 \vee \bar{x}_2 \bar{x}_4 x_8 \vee \bar{x}_2 x_6 x_8 \vee x _4 x_6 x_8$$
Теперь запишем, что получилось:
$$(\bar{x}_1 \bar{x}_3 \bar{x}_5 \bar{x}_7 \vee \bar{x}_1 \bar{x}_3 \bar{x}_5 x_9 \vee\bar{x}_1 \bar{x}_3x_7 x_9 \vee \bar{x}_1 x_5 x_7 x_9 ~ \vee$$
$$\vee ~ x_3 x_5 x_7 x_9)(\bar{x}_2 \bar{x}_4 \bar{x}_6 \vee \bar{x}_2 \bar{x}_4 x_8 \vee \bar{x}_2 x_6 x_8 \vee x _4 x_6 x_8)$$
Раскрыв скобки, получаем:
$$x_1\bar{x}_2 x_3 \bar{x}_4 x_5 \bar{x}_6 x_7 x_9 \vee x_1 \bar{x}_2 x_3 \bar{x}_4 x_5 x_7 x_8 x_9 \vee x_1 \bar{x}_2 x_3 x_5 x_6 x_7 x_8 x_9 ~ \vee$$ 
$$\vee ~ x_1 x_3 x_4 x_5 x_6 x_7 x_8 x_9 \vee \bar{x}_1 x_2 \bar{x}_3 x_4 \bar{x}_5 x_6 \bar{x}_7 x_8 \vee \bar{x}_1 x_2 \bar{x}_3 x_4 \bar{x}_5 x_6 x_8 x_9 ~\vee $$ 
$$\vee ~ \bar{x}_1 x_2 \bar{x}_3 x_4 x_6 x_7 x_8 x_9 \vee \bar{x}_1 x_2 x_4 x_5 x_6 x_7 x_8 x_9 \vee x_2 x_3 x_4 x_5 x_6 x_7 x_8 x_9$$
Для относительного удобства приведем к СДНФ:
$$\bar{x}_1 x_2 \bar{x}_3 x_4 \bar{x}_5 x_6 \bar{x}_7 x_8 \bar{x}_9 \vee \bar{x}_1 x_2 \bar{x}_3 x_4 \bar{x}_5 x_6 \bar{x}_7 x_8 x_9 ~ \vee$$
$$\vee~\bar{x}_1 x_2 \bar{x}_3 x_4 \bar{x}_5 x_6 x_7 x_8 x_9 \vee \bar{x}_1 x_2\bar{x}_3 x_4 x_5 x_6 x_7 x_8 x_9 ~\vee$$
$$\vee~\bar{x}_1 x_2 x_3 x_4 x_5 x_6 x_7 x_8 x_9 \vee x_1 \bar{x}_2 x_3 \bar{x}_4 x_5 \bar{x}_6 x_7 \bar{x}_8 x_9~\vee$$
$$\vee~x_1 \bar{x}_2 x_3 \bar{x}_4 x_5 \bar{x}_6 x_7 x_8 x_9 \vee x_1 \bar{x}_2 x_3 \bar{x}_4 x_5 x_6 x_7 x_8 x_9~\vee$$
$$\vee~x_1 \bar{x}_2 x_3 x_4 x_5 x_6 x_7 x_8 x_9 \vee x_1 x_2 x_3 x_4 x_5 x_6 x_7 x_8 x_9$$
\newline \newline
 \noindent \textbf{Задача 5}
\begin{center}
\textit{Докажите полноту системы связок, состоящей из одной связки Штрих Шеффера $x|y=\neg(x\wedge y)$} 
\end{center}
Воспользуемся полнотой системы $\{\vee, \neg\}$. Достаточно привести аналоги данным операциям:
$$\bar{x}: (x \Leftrightarrow x \wedge x) \Longleftrightarrow (\overline{x} \Leftrightarrow \overline{x \wedge x}) \Longleftrightarrow (\overline{x} \Leftrightarrow (x~|~x))$$
$$x \vee y : (x \vee y \Leftrightarrow \overline{(\bar{x}\wedge \bar{y}))} \Longleftrightarrow (x|x)|(y|y) $$
Заменяя операции системы $\{\vee, \neg\}$, мы получаем, что можно выразить любое высказывание через связки Штриха Шеффера.
\newline \newline
 \noindent \textbf{Задача 6}
\begin{center}
\textit{КНФ (конъюнктивной нормальной формой) называется конъюнкция дизъюнкций переменных или их отрицаний. Докажите, что любое высказывание можно выразить в виде КНФ.}
\end{center}
Воспользуемся таким утверждением, что любое высказывание можно выразить в виде ДНФ. Возьмем произвольное высказывание $X$. Приведем отрицание этого высказывания $\overline{X}$ (оно также имеет ДНФ) к ДНФ (обозначим результат как $X'$) и снова применим к нему отрицание, воспользовавшись законом Де Моргана:
$$\overline{\left (~\bigvee_{i=0}^{n} ~ X'_i~\right)} \Longleftrightarrow \left (~\bigwedge_{i=0}^{n} ~ \overline{X'}_i~\right)$$
Так как $$\overline{X'}\Leftrightarrow\overline{\overline{X}}\Leftrightarrow X$$
Получаем,что произвольное высказывание $X$ мы можем привести к КНФ 
\newline
\newline \newline
 \noindent \textbf{Задача 7}
\begin{center}
\textit{Сколько ненулевых коэффициентов в многочлене Жегалкина, который равен $x_1\vee x_2\vee..\vee x_n?$}
\end{center}
Докажем по индукции, что ненулевых коэффициентов будет $2^n - 1$
База индукции $n = 1$: 
Ненулевых коэффициентов получаем 1, что соответствует предположению индукции \newline
Индукционный шаг $n \rightarrow n+1$:
$$(x_1\vee x_2\vee..\vee x_n \vee x_{n+1}) =  (x_1\vee x_2\vee..\vee x_n) \oplus x_{n+1} \oplus x_{n+1}\wedge (x_1\vee x_2\vee..\vee x_n)$$
Каждое из этих слагаемых будет ненулевым, по предположению индукции получаем:
$$(2^n-1) + 1 + (2^n - 1) = 2 \cdot 2^n -1 = 2^{n+1}-1$$
Что и требовалось доказать.
\newline
\newline 
\noindent \textbf{Задача 8}
\begin{center}
\textit{Будет ли полной система \{$\neg , MAJ(x_1, x_2, x_3)$\}?}
\end{center}
Пусть система \{$\neg , MAJ(x_1, x_2, x_3)$\} является полной. Тогда мы можем выразить конъюнкцию двух переменных $x_1, x_2$. Пусть $x_1 = 0, x_2 = 1$, а $x_1 \wedge x_2 = 0$. Теперь рассмотрим случай, когда мы проведем инверсию значений всех переменных. Тогда $x_1 = 1, x_2 = 0$. Кроме того, достаточно легко заметить, что тогда и изменится значение всех $MAJ(x_1, x_2, x_3)$, участвующих в представлении конъюнкции, на противоположное. Соответственно и изменится значение всех $\neg$ на противоположное. Получаем, что при инверсии переменных инвертируется и само значение высказывания. (Индукция по вложенным операциям). Таким образом, получаем, что $x_1 \wedge x_2 = 1$. Противоречие.
\newline
 \newline
\end{document}
