\documentclass{article}     
\usepackage[utf8]{inputenc} 
\usepackage[T2A]{fontenc}
\usepackage{amsmath}
\title{Домашнее задание по дискретной математике}  
\author{Родигина Анастасия, 167 группа}     
\date{25 января 2017}   

\usepackage{graphicx}
\graphicspath{{pictures/}}
\DeclareGraphicsExtensions{.pdf,.png,.jpg}

\newcommand{\ip}[2]{(#1, #2)}
                             
\begin{document}            

\maketitle  
 \noindent \textbf{Задача 1}
\begin{center} 
\textit{Верно ли что множество всех кругов на плоскости имеет мощность континуум?}
\end{center}
Заметим, что можно построить биекцию между множеством всех кругов на плоскости и множеством троек действительных чисел $R^3$. Каждый круг будет задаваться единственным образом тремя действительными числами: координатами центра круга и радиусом (Радиус определен на $R_+$, однако использум тот факт, что $R_+ \sim R$). Соответственно одна тройка будет задавать только один круг. Пусть $A$ - множество всех кругов на плоскости. Используем теорему Кантора-Бернштейна. Так как мы можем построить биекцию между множеством всех кругов на плоскости и множеством троек действительных чисел $R^3$: $$A \sim R^3$$ 
Зная, что: $$R^k \sim R$$
Получаем: $$A \sim R^3 \sim R$$
Тогда множество всех кругов на плоскости действительно имеет континуальную мощность.
\newline
 \noindent \textit{Ответ: Да, верно.}
\newline
\newline
\textbf{Задача 2}
\begin{center}
\textit{На плоскости отмечено континуум окружностей. Верно ли, что множество их центров имеет мощность континуум?}
\end{center}
Приведем такой контрпример, когда на плоскости отмечено континуум окружностей, однако множество их центров конечно. 
Зафиксируем какую-то точку и проведем все возможные окружности с центром в этой точке. Множество всех возможных радиусов будет равномощно $R_+ \sim R$, следовательно, на плоскости будет отмечено континуум окружностей, а множество их центров будет конечно (будет иметь мощность 1).
\newline
 \noindent \textit{Ответ: Нет, неверно.} 
 \newline
 \newline
 \newline
 \newline
\textbf{Задача 3}
\begin{center}
\textit{Существует ли континуальное семейство непересекающихся континуальных подмножеств?} 
\end{center}
Приведем такой пример: на координатной плоскости построим квадрат с углом в точке $(0,0)$ и стороной $1$. Внутри этого квадрата проведем все возможные отрезки c с левым концом на оси ординат, параллельные оси абсцисс.Таких отрезков внутри квадрата будет континуальное множество, так как их множество их левых концов равномощно (0, 1) а, значит, и равномощно $R$. Сами эти отрезки тоже будут континуальными множествами действительных чисел на отрезке [0,1] (Можно провести биекцию, достаточно взять действительные координаты каждой точки). Каждая пара координат внутри квадрата будет элементом континуального множества, а этих множеств также континуум. Кроме того, эти множества не пересекаются. Таким образом, получаем континуальное семейство (множество) непересекающихся континуальных подмножеств)
\begin{center}
 \includegraphics[width=120pt, height = 120pt]{z4.png}
\end{center}
\noindent \textit{Ответ: Да, существует.}
\newline
\newline
 \noindent \textbf{Задача 4}
\begin{center}
\textit{Верно ли, что множество бесконечных двоичных последовательностей, в которых нет трех единиц подряд, имеет мощность континуум?}
\end{center}
Известно, что множество бесконечных двоичных последовательностей является континуальным. Построим биекцию между этим множеством и множеством бесконечных двоичных последовательностей, в которых нет трех единиц подряд. В каждой двоичной последовательности из первого множества поставим по 0 после каждых двух идущих подряд единиц. Таким образом, мы перейдем из первого множества ко второму. Заметим, что данная операция определяет инъективную функцию. Докажем это от обратного, предположим, что какие-то две разных двоичных последовательности этой операцией переводятся в последовательность из нашего множества. Очень важно заметить, что обратная операция (стереть 0 после каждых 2 идущих подряд единиц) однозначна. Т.е. убрав добавленные нули, мы получаем двоичную последовательность однозначно. Получаем противоречие. Таким образом, мы показали инъективность. Легко заметить, что получаемая функция сюръективна. Получаем биекцию.
\newline
\noindent \textit{Ответ: Да, верно.}
\newline \newline
 \noindent \textbf{Задача 5}
\begin{center}
\textit{Докажите, что множество биекций $N\mapsto N$ имеет мощность континуум.} 
\end{center}
Покажем, что множество таких биекций не более, чем континуально. Каждую биекцию $N\mapsto N$ запишем следующим образом: запишем последовательность натуральных чисел. Каждому числу, стоящему на нечетном месте, будет взаимно однозначно соответствовать следующее число, а числу, стоящему на четном месте - предыдущее.Теперь проведем инъекцию из множества биекций в множество бесконечных двоичных последовательностей. Каждое число последовательности $a_i$ запишем в виде $a_i$ единиц, чила будем разделять одним нулем. (Например последовательность 4, 2, 1... запишется как 111101101...). По Теореме, доказанной на семинаре получаем, что мощность множества биекций $N \mapsto N$ не больше мощности множества бесконечных двоичных последовательностей - континуума. 
Теперь проиллюстрируем, что множество таких биекций не менее, чем континуально. Проведем инъекцию из множества бесконечных двоичных последовательностей в множество биекций $N\mapsto N$. Разобъем натуральные числа на пары соседних (0 и 1, 2 и 3...) и пронумеруем эти пары. Каждой последовательности нулей и единиц сопоставим последовательность неповторяющихся натуральных чисел (последовательность будет задавать биекцию, как в первой части решения) таким образом: если на i-м месте стоит 0, то порядок чисел в паре меняться не будет, если 1, то меняем их порядок и записываем в последовательность. Так как у нас получается построить инъекцию, получаем, что можность биекций $N\mapsto N$ не менее, чем континуально. По Теореме Кантора-Бернштейна получаем, что множество биекций $N\mapsto N$ имеет мощность континуум.
\newline \newline
 \noindent \textbf{Задача 6}
\begin{center}
\textit{Можно ли расположить на плоскости континуум непересекающихся равных единиц? (Единицами называются фигуры, изображенные на рисунке, т.е. пары отрезков с общим концом.)}
\end{center}
Расположим все единицы таким образом: пусть их соответствующие отрезки лежат параллельно друг другу (таким образом, они не будут пересекаться), а их концы (уголки) лежат на одной прямой. Каждой единице будет взаимно однозначно соответствовать координата "уголка" на прямой. Заметим, что множество таких единиц будет равномощно множеству вещественных чисел (можно провести биекцию между любым действительным числом и единичкой, у которой уголок будет лежать на соответствующей координате на прямой действительных чисел), т. е. количество таких единиц будет континуально.
\newline
\begin{center}
 \includegraphics[width=120pt, height = 80pt]{z6.png}
\end{center}
\noindent \textit{Ответ: Да, можно.}
\newline \newline
 \noindent \textbf{Задача 7}
\begin{center}
\textit{Можно ли расположить на плоскости континуум непересекающихся восьмерок. (Восьмерка - это объединение касающихся внешним образом окружностей.)}
\end{center}
Заметим, что между множеством таких восьмерок и множеством рациональных чисел можно провести инъекцию.Каждой такой восьмерке сопоставим две пары рациональных чисел - произвольные рациональные координаты, находящиеся внутри каждой окружности. Покажем существование этих рациональных координат. Внутри окружности посторим квадрат, стороны которого параллельны осям координат. Используя факт, что на любом интервале содержится рациональное число, берем сторону квадрата и находим на нем рациональное число, записываем как координату, повторяем то же самое с прилежащей стороной квадрата. Таким образом мы можем поставить в соответствие с каждой восьмеркой две пары координат в рациональных числах. Эти пары будут однозначно задавать только одну восьмерку, тк иначе получаем противоречие отсутствию пересечений восьмерок. Построив инъекцию между множеством таких восьмерок и множеством рациональных чисел, получаем, что множество восьмерок не более, чем счетно. Мощность рациональных чисел меньше континуума, а, значит, на плоскости расположить континуум непересекающихся восьмерок мы не сможем.
 \newline
 \begin{center}
 \includegraphics[width=120pt, height = 120pt]{z7.png}
\end{center}
\noindent \textit{Ответ: Нет, нельзя.}
\end{document}
