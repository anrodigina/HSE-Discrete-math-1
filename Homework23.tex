\documentclass{article}     
\usepackage[utf8]{inputenc} 
\usepackage{amsfonts}
\usepackage[left=4cm,right=4cm,
    top=3cm,bottom=4cm,bindingoffset=0cm]{geometry}
\sloppy
\usepackage[T2A]{fontenc}
\usepackage{amsmath}
\title{Домашнее задание по дискретной математике}  
\author{Родигина Анастасия, 167 группа}     
\date{16 марта  2017}   

\usepackage{graphicx}
\graphicspath{{pictures/}}
\DeclareGraphicsExtensions{.pdf,.png,.jpg}

\newcommand{\ip}[2]{(#1, #2)}
                             
\begin{document}            

\maketitle  
 \noindent \textbf{Задача 1}
\begin{center} 
\textit{Постройте МТ, которая вычисляет нигде не определенную функцию.}
\end{center}
МТ вычисляет нигде не определенную функцию тогда, когда она не завершает свою работу. 
Зададим такую МТ:
Алфавит: $\{a,\Lambda\}$, где $a$ - произвольный символ
$$\delta(a,0)=(a,0,0)$$
$$\delta(\Lambda,0)=(\Lambda,0,+1)$$
Достаточно заметить, что при любом входе МТ не закончит свою работу. (Так как финальное состояние не определено).
\newline
\newline
\textbf{Задача 2}
\begin{center}
\textit{Постройте МТ, которая инвертирует входное двоичное слово: на входе $w$, где $w = w_1 . . . w_n, w_i \in
\{0, 1\}$, результатом работы должно быть слово $w = \overline{w}_1 . . . \overline{w}_n$.}
\end{center}
Зададим МТ:
$$\delta(1,0)=(0,0,+1)$$
$$\delta(0,0)=(1,0,+1)$$
$$\delta(\Lambda,0)=(\Lambda,1,-1)$$
$$\delta(1,1)=(1,1,-1)$$
$$\delta(0,1)=(0,1,-1)$$
$$\delta(\Lambda,1)=(0,42,+1)$$
 Докажем корректность. Конфигурация $0w$ переходит в конфигурацию $\overline{w}0$, если выполняются первые две строчки. Выполнение третьей строки переводит конфигурацию в $\overline{w}_0...\overline{w}_{n-1}1\overline{w}_n$. Выполнение четвертой и пятой строчки приведет к конфигурации $1\Lambda\overline{w}$. Затем МТ перейдет в финальное состояние и закончит работу.  
 \newline
 \newline
\textbf{Задача 3}
\begin{center}
\textit{Постройте МТ, проверяющую, входит ли в слово в алфавите $\{a, b, c\}$ подслово $aba$. В конце работы
машины на ленте должно остаться 1, если такое подслово есть и 0, если его нет.} 
\end{center}
$$\delta(a,0)=(\Lambda,1,+1)$$
$$\delta(b,0)=(\Lambda,0,+1)$$
$$\delta(c,0)=(\Lambda,0,+1)$$

$$\delta(a,1)=(\Lambda,1,+1)$$
$$\delta(b,1)=(\Lambda,2,+1)$$
$$\delta(c,1)=(\Lambda,0,+1)$$

$$\delta(a,2)=(\Lambda,3,+1)$$
$$\delta(b,2)=(\Lambda,0,+1)$$
$$\delta(c,2)=(\Lambda,0,+1)$$

$$\delta(a,3)=(\Lambda,3,+1)$$
$$\delta(b,3)=(\Lambda,3,+1)$$
$$\delta(c,3)=(\Lambda,3,+1)$$

$$\delta(\Lambda,3)=(1,42,+1)$$
$$\delta(\Lambda,0)=(0,42,+1)$$
$$\delta(\Lambda,1)=(0,42,+1)$$
$$\delta(\Lambda,2)=(0,42,+1)$$

Докажем корректность. Ищем первое вхождение символа $a$. Если его находим, меняем состояние на 1, проверяем следующий символ, если он равен а, то не меняем состояние, если он равен в, то меняем состояние на 2, иначе меняем состояние на ноль. Если мы находимся в состоянии 2, то проверяем следующий символ, если он равен а, то меняем состояние на 3, иначе возвращаем состояние в ноль. Заметим, что подстрока находится тогда и только тогда, когда мы приходим к состоянию три, а потом записываем 1, если это так. 
\newline
\newline
 \noindent \textbf{Задача 4}
\begin{center}
\textit{Докажите существование МТ, которая сортирует символы входного двоичного слова: на входе w,
где двоичное слово w содержит a нулей и b единиц, результатом работы должно быть слово $0^a1^b$.}
\end{center}
Опишем алгоритм работы МТ. Проходим по ленте МТ, если встречаем 0, то переходим на одну клетку направо, все это время находимся в состоянии 0. Когда мы встречаем 1, то меняем эту 1 на $\Lambda$ и меняем состояние на 1. Затем идем вперед по ленте не меняя ничего до тех пор пока не встретим 0, тогда меняем ноль на 1 и меняем состояние на 2, будем двигаться назад до тех пор пока не встретим $\Lambda$, меняем ее на 0, а состояние меняем на 0. Рассмотрим такой случай, когда в состоянии 1 мы не находим 0 до состояния $\Lambda$, тогда мы вернемся назад, поменяем $\Lambda$ на 1 и завершим работу. Если мы доходим до $\Lambda$ в состоянии 0, то завершаем работу. Можно заметить, что результатом работы этой программы будет строка из $a$ нулей и $b$ единиц. 
\newline \newline
 \noindent \textbf{Задача 5}
\begin{center}
\textit{Докажите существование МТ, которая проверяет, что вход является палиндромом. (Слово $a_1a_2 . . . a_n$ называется палиндромом, если $a_1a_2 . . . a_n = a_na_{n-1} . . . a_1$.) Если вход является палиндромом, результат работы должен быть 1, а если нет, то результат 0.} 
\end{center}
Возьмем МТ с двумя лентами. Лентой входа будет первая лента, лентой результата тоже будет первая лента. Скопируем входное слово на вторую ленту, каретка на второй ленте будет указывать на конец. Переведем на первой ленте каретку на первый символ. Будем сравнивать элементы на двух лентах (с помощью таблицы переходов). Если равенство нарушается, поменяем состояние, когда дойдем до символа конца ленты, в зависимости от состояния запишем наш результат на первой ленте, в технические детали вдаваться не будем, это сделать достаточно несложно. Таким образом, мы осуществили  проверку на палиндром.
\newline \newline
 \noindent \textbf{Задача 6}
\begin{center}
\textit{Существует ли машина Тьюринга, при начале работы на пустой ленте, оставляющая на ней 2017 единиц и имеющая не более 100 состояний?}
\end{center}
Возьмем машину Тьюринга с 2018 лентами, первая лента будет лентой входа и результата одновременно. На всех лентах (кроме первой) напишем 0 - в нулевом состоянии. В первом состоянии будем писать 1 на первой ленте и будем сдвигать вправо, а потом на следующей ленте, на которой мы не встречаем $\Lambda$, вместо 0 пишем $\Lambda$. (Это задается таблицей перехода). Если во всех лентах (кроме первой) будет стоять $\Lambda$, то переходим в финальное состояние и на первой ленте пишем $\Lambda$. Таким образом, мы построили машину Тьюринга, при начале работы на пустой ленте, оставляющей на ней 2017 единиц и имеющую не более 100 состояний.
\newline
\newline
 \noindent \textbf{Задача 7}
\begin{center}
\textit{Докажите существание машины Тьюринга, вычисляющей какую-либо биекцию между $\mathbb{N} \times  \mathbb{N}$ и $\mathbb{N}$. Т.е. в начале работы на ленте $1^a\#1^b$, а в конце $1^{f(a,b)}$, где $f$   выбранная Вами биекция $f$.}
\end{center}
Воспользуемся явной биекцией между $\mathbb{N} \times  \mathbb{N}$ и $\mathbb{N}$: $(x,y) \mapsto \binom{x+y+1}{2}+y$. Она является вычислимой (это обсуждалось на семинаре) и по тезису Черча-Тьюринга может быть вычислима на МТ. 
\newline
\newline 
\end{document}
