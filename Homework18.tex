\documentclass{article}     
\usepackage[utf8]{inputenc} 
\usepackage{amsfonts}
\usepackage[left=4cm,right=4cm,
    top=3cm,bottom=4cm,bindingoffset=0cm]{geometry}
\sloppy
\usepackage[T2A]{fontenc}
\usepackage{amsmath}
\title{Домашнее задание по дискретной математике}  
\author{Родигина Анастасия, 167 группа}     
\date{8 февраля 2017}   

\usepackage{graphicx}
\graphicspath{{pictures/}}
\DeclareGraphicsExtensions{.pdf,.png,.jpg}

\newcommand{\ip}[2]{(#1, #2)}
                             
\begin{document}            

\maketitle  
 \noindent \textbf{Задача 1}
\begin{center} 
\textit{Докажите, что функцию $x \oplus y \oplus z$ можно вычислить схемой, использующей лишь одно отрицание (и много конъюнкций и дизъюнкций).}
\end{center}
Заметим, что $x \oplus y \oplus z$ - это сложение по модулю два, таким образом, высказывание будет принимать истину только тогда, когда 1 в строке таблицы истинности нечетное количество. 
Запишем ДНФ:
$$(xy\bar{z} \vee x\bar{y}z \vee \bar{x}yz) \vee xyz$$
Попробуем выразить те строки таблицы, когда количество 1 равно 1, через функцию $MAJ(x_1,x_2,x_3)$, которую можно также выразить через конъюнкции и дизъюнкции. 
Получаем такое высказывание (учитывая ситуацию, когда все переменные имеют значение 1):
$$f = (x_1\vee x_2 \vee x_3)\wedge \overline{MAJ(x_1,x_2,x_3)}\vee xyz$$
Проверим это высказывание таблицей истинности:
\begin{center}
\begin{tabular}{|c|c|c|c|c|}
\hline
$x_1$ & $x_2$ & $x_3$ & $x_1 \oplus x_2 \oplus x_3$ & $f$  \\
\hline
0  & 0 & 0 & 0 & 0 \\
\hline
0  & 0 & 1 & 1 & 1  \\
\hline
0  & 1 & 0 & 1 & 1 \\
\hline
0  & 1 & 1 & 0 & 0 \\
\hline
1  & 0 & 0 & 1 & 1 \\
\hline
1  & 0 & 1 & 0 & 0 \\
\hline
1  & 1 & 0 & 0 & 0 \\
\hline
1  & 1 & 1 & 1 & 1 \\
\hline

\end{tabular}
\end{center}
Теперь выразим MAJ через конъюнкции и дизъюнкции:
$$MAJ(x_1,x_2,x_3) = ((x_1\vee x_2)\wedge x_3) \vee (x_1\wedge x_2)$$
Теперь запишем функцию $x \oplus y \oplus z$:
$$(x_1\vee x_2 \vee x_3)\wedge \overline{((x_1\vee x_2)\wedge x_3) \vee (x_1\wedge x_2)}\vee x_1 x_2 x_3$$
Так как эту функцию можно записать, используя лишь одно отрицание и много дизъюнкций и конъюнкций, ее можно вычислить соответствующей схемой.
 \begin{center} 
 \includegraphics[width=150pt, height = 150pt]{Scg.png}
 \end{center}
Размер схемы равен 12
\newline
\newline
\textbf{Задача 2}
\begin{center}
\textit{Функция $f (x_1, x_2, x_3, x_4)$ истнинна на последних 9 наборах значений переменных (в стандартном порядке) и только на них. Постройте схему, вычисляющую $f$, использующую только дизъюнкцию и конъюнкцию длины не более 15.}
\end{center}
Заметим, что 7 строка в таблице истинности - 0111, а все для всех последующих выполняется условие $x_1=1$. Таким образом, можно записать функцию в виде:
$$x_1 \vee x_2 x_3 x_4$$
Построим схему:
 \begin{center} 
 \includegraphics[width=150pt, height = 150pt]{scheme2.png}
 \end{center}
 Размер схемы будет равен 6
 \newline
 \newline
\textbf{Задача 3}
\begin{center}
\textit{Постройте схему полиномиального размера, проверяющую, что во входное слово входит подслово 101. Можно считать, что длина входного слова не меньше трех. } 
\end{center}
Запишем эту схему полиномиального размера в строку:
$$x_1, x_2,..,x_n;~ \bar{x}_1, \bar{x}_2, \bar{x}_3;~ (x_1 \bar{x}_2 x_3), (x_2 \bar{x}_3 x_4), ..,(x_{n-2},\bar{x}_{n-1},x_n); ~(x_1 \bar{x}_2 x_3)\vee (x_2 \bar{x}_3 x_4)\vee .. \vee (x_{n-2},\bar{x}_{n-1},x_n)$$
Оценим эту размер этой схемы как $O(n)$, так как при каждом уровне мы совершаем не более, чем $n$ операций, а уровней конечное количество (4).
\newline
\newline
 \noindent \textbf{Задача 4}
\begin{center}
\textit{Постройте схему полиномиального размера, умножащую двоичное число на три}
\end{center}
Запишем схему для операции $x \oplus y$, чтобы потом упростить итоговую схему:
$$x, y; \bar{x}, \bar{y}; x\wedge \bar{y}, y\wedge \bar{x}; x\bar{y}\vee \bar{x}y$$
Аналогично запишем схему для $MAJ(x,y,z)$:
$$x, y, z;~ x\vee y, x \wedge y;~ (x\vee y) \wedge z;~((x\vee y) \wedge z) \vee (x \wedge y) $$
Определим схему для сложения чисел в двоичной записи. Запишем первое число как $x=x_n x_{n-1} x_{n-2}..x_0$, а второе $z=z_n z_{n-1} z_{n-2}..z_0$ (При необходимости для совпадений длины можно добавить ведущие нули). Сумму обозначим за $s = s_{n+1} s_n .. s_0$ (Заметим, что операция умножения на три соответствует дважды проведенной операции сложения чисел).  Обозначим за $c_i$ разряд переноса. Заметим, что $c_0 = 0; c_i = MAJ (x_{i-1}, z_{i-1}, c_{i-1})$, а $s_i = x_i \oplus z_i \oplus c_i$ Таким образом можно построить схему для вычисления суммы двух двоичных чисел. К исходному числу добавляем исходное дважды (При необходимости снова можно будет добавить ведущие нули). Получаем умножение на 3, дублируя части схемы.
Оценим размер схемы как $O(n)$, так как для каждого разряда суммы совершаем конечное количество операций, а сумму вычисляем дважды.
\newline \newline
 \noindent \textbf{Задача 5}
\begin{center}
\textit{Постройте схему полиномиального размера, проверяющую, будет ли $n$-битное двоичное число делиться на 3.} 
\end{center}
Заметим, что степени 2 могут давать два остатка при делении на 3: 1 или 2. Кроме того, эти остатки будут чередоваться в зависимости от четности показателя степени 2, 1 - если степень четная и 2 - если степень нечетная. Запишем исходное число как $x=\overline{x_n x_{n-1}..x_0}_2=x_0 + x_1 \cdot 2 + x_2 \cdot 2^2+..+x_n \cdot 2^n $
Заметим, что
$$x = x_0 + x_1 \cdot 2 + x_2 \cdot 2^2+..+x_n \cdot 2^n \equiv x_0 + x_1 \cdot 2 + x_2+ x_3\cdot 2+...~ (mod ~3)$$
$$x_0 + x_1 \cdot 2 + x_2+ x_3\cdot 2+..+x_n \cdot 2^{n} \equiv x_0 - x_1 + x_2 - x_3+..+(-1)^{n-1}x_n(mod ~3)$$
Организуем нашу схему таким образом, в переменных $z_2, z_1, z_0$ будем хранить остаток по модулю три при каждой итерации. Сделаем так, чтобы одна и только одна из переменных принимала истинное значение, остаток по модулю три будет соответствовать ее индексу. Обозначим за $x_+$ и $x_-$, значения $x$ с четным индексом и нечетным соответственно.
Рассмотрим ситуацию, когда $x_+=1$, построим таблицу истинности, если $x_+=0$, то переменные будут оставлять свои старые значения:
\begin{center}
\begin{tabular}{|c|c|c|c|c|c|c|}
\hline
$z_2$ & $z_1$ & $z_0$ & $\Rightarrow$ & $z_2$ & $z_1$ & $z_0$   \\
\hline
0 & 0 & 1 & $\Rightarrow$  & 0 & 1 & 0 \\
\hline
0 & 1 & 0 & $\Rightarrow$  & 1 & 0 & 0 \\
\hline
1 & 0 & 0 & $\Rightarrow$  & 0 & 0 & 1 \\
\hline
\end{tabular}
\end{center}
Выразим $z_2, z_1, z_0$, используя $x_+$:
$$z_2 = \bar{z}_2 z_1 \bar{z}_0 \wedge x_+ \vee z_2 \wedge \bar{x}_+$$
$$z_1 = \bar{z}_2 \bar{z}_1 z_0 \wedge x_+ \vee z_1 \wedge \bar{x}_+$$
$$z_0 = z_2 \bar{z}_1 \bar{z}_0 \wedge x_+ \vee z_0 \wedge \bar{x}_+$$

Аналогично рассмотрим ситуацию с $x_-$
\begin{center}
\begin{tabular}{|c|c|c|c|c|c|c|}
\hline
$z_2$ & $z_1$ & $z_0$ & $\Rightarrow$ & $z_2$ & $z_1$ & $z_0$   \\
\hline
0 & 0 & 1 & $\Rightarrow$  & 1 & 0 & 0 \\
\hline
0 & 1 & 0 & $\Rightarrow$  & 0 & 0 & 1 \\
\hline
1 & 0 & 0 & $\Rightarrow$  & 0 & 1 & 0 \\
\hline
\end{tabular}
\end{center}
Выразим $z_2, z_1, z_0$, используя $x_-$:
$$z_2 = \bar{z}_2 \bar{z}_1 z_0 \wedge x_- \vee z_2 \wedge \bar{x}_- $$
$$z_1 = z_2 \bar{z}_1 \bar{z}_0 \wedge x_- \vee z_1 \wedge \bar{x}_- $$
$$z_0 = \bar{z}_2 z_1 \bar{z}_0 \wedge x_- \vee z_0 \wedge \bar{x}_- $$
Установим первоначальное значение для $z$: $z_2=0, z_1=0, z_0=1$
Будем чередовать части схемы для четных и нечетных $x$ 
На выходе делимость на 3 проверяется значением $z_0$.
Оценим размер схемы как $O(n)$. Каждой части схемы (для $x$ с идексами различной четности) будет соответствовать конечное число операций, таких частей схемы $n$.  
\newline
\newline

\end{document}
