\documentclass{article}     
\usepackage[utf8]{inputenc} 
\usepackage[T2A]{fontenc}
\usepackage{amsmath}
\title{Домашнее задание по дискретной математике}  
\author{Родигина Анастасия, 167 группа}     
\date{18 января 2017}   

\newcommand{\ip}[2]{(#1, #2)}
                             
\begin{document}            

\maketitle  
 \noindent \textbf{Задача 1}
\begin{center} 
\textit{Верно ли, что если $A\setminus B$ бесконечно, а $B$ счетно,
     то $A\setminus B$ равномощно $A$?}
\end{center}
\indent Представим множество $A$ в виде:
$$A = (A \setminus B) \cup (A \cap B)$$
Учитывая то, что любое подмножество счетного множества, счетно, заметим, что $(A \cap B) \subseteq B$ также счетно (или конечно).
Так как $(A \setminus B)$ бесконечно, а $(A \cap B)$ счетно (или конечно), то (по Теореме)
$$(A \setminus B) \cup (A \cap B) \sim (A \setminus B) $$
А следовательно $$A \sim (A \setminus B)$$
\textit{Ответ: Да, верно.}
\newline
\newline
\textbf{Задача 2}
\begin{center}
\textit{ Верно ли, что если $A$ бесконечно, а $B$ счетно, то $A \bigtriangleup B$} равномощно $A$?
\end{center}

Достаточно привести контрпример, для того, чтобы показать, что это утверждение неверно.
Возьмем, что $A = B$, где $A$ и $B$ непустые множества
Тогда получчаем, что $A \bigtriangleup B = \emptyset$, что неэквивалентно $ A$ \newline
\textit{Ответ: Нет, неверно.} \newline
\\
\textbf{Задача 3}
\begin{center}
\textit{Верно ли что если $A$ бесконечно, а $B$ конечно, то $(A \setminus B)$ равномощно $A$?} 
\end{center}
Так как $A$ бесконечно, $B$ конечно, $(A \setminus B)$ равномощно $A$. Представим множество $A$ в виде:
$$A = (A \setminus B) \cup (A \cap B)$$
$ (A \setminus B)$ будет бесконечным, а $A \cap B$ будет счетным.
Таким образом получаем, что 
$$(A \setminus B) \cup (A \cap B) \sim (A \setminus B)$$
$$ A \sim (A \setminus B)$$
\textit{Ответ: Да, верно.}
\newline
\newline
\textbf{Задача 4}
\begin{center}
\textit{Докажите, что любое множество непересекающихся интервалов на прямой конечно или счетно.}
\end{center}
Заметим, что в любом непустом интервале содержится рациональное число. Если оба конца интервала являются рациональными, то достаточно взять их среднее арифметическое, которое будет лежать внутри этого интервала и будет также рациональным числом. Если хотя бы один из концов интервала является иррациональным, то достаточно использовать приближение этих чисел снизу или (сверху с точностью до $\frac{1}{10^{n+1}}$, где $10^{n}<|a-b|$, a,b - концы интервала).
Так как интервалы не пересекаются, то мы можем построить инъекцию из мн-ва интервалов в  подмножество рациональных чисел, которые содержатся внутри этих интервалов. Так как множество рациональных чисел счетно, любое его подмножество будет являться либо конечным, либо счетным.
\newline \newline
\textbf{Задача 5}
\begin{center}
\textit{Докажите, что всякое бесконечное множество содержит бесконечное число непересекающихся счетных подмножеств} 
\end{center}
Используем теорему, что из любого бесконечного множества выделить счетное подмножество. По определению счетного множества, мы можем провести биекцию между этим множеством и $N$. В этом счетном подмножестве выделим два подмножества (также счетных), одно из них будет с четными номерами, другое с нечетными. Снова проведем биекцию и. т. д. Мы можем повторить эти действия бесконечное количество раз, так как на каждой итерации новое подмножество остается бесконечным и счетным. Таким образом, всякое бесконечное множество содержит бесконечное число непересекающихся счетных подмножеств
\newline \newline
\textbf{Задача 6}
\begin{center}
\textit{Функция называется периодической, если для некоторого числа $T$ и любого $x$ выполняется $f(x+T)~=~f(x)$. Докажите, что множество периодических функций счетно.}
\end{center}
Покажем, что множество таких функций не менее, чем счетно. Для этого достаточно рассмотреть множество функций с периодом 1. Каждому целому числу будет соответствовать одна и только одна такая функция, таким образом множество таких функций (постоянных) будет счетно, а значит множество всех периодических функций будет не менее, чем счетно.
Рассмотрим произвольную функцию с каким-то целым периодом $T$. Заметим, что достаточно ей сопоставить последовательность из $T$ (от $0$ до $T-1$) целых чисел, где $a_{\textit{i}}=f(i ~ mod ~ T)$, где $i \in {Z}$ и $i \in [0,~T-1]$. Причем для разных функций эта последовательность разная. Можно заметить, что мы получаем инъекцию в множество целых чисел. Известно, что множество всех последовательностей целых чисел, счетно. Следовательно, наше множество не более, чем счетно. Таким образом, получаем, что множество периодических функций счетно.
\newline \newline
\textbf{Задача 7}
\begin{center}
\textit{Постройте явную биекцию между конечными строго возрастающими последовательностями натуральных чисел и конечными последовательностями натуральных чисел.}
\end{center}
Заметим, что посторение биекции возможно тогда и только тогда, когда количество членов в обоих последовательностях одинаково. Пусть $n$ - длина одной из них. Если $n\leq1$, то достаточно просто потребовать, чтобы $a_{\textit{i}} = b_{\textit{i}}$. Теперь рассмотрим вариант, когда $n\geq2$.

\begin{equation*}
f : 
 \begin{cases}
   b_{\textit{1}} = a_{\textit{1}} &\text{i = 1}\\
   b_{\textit{i}} = a_{\textit{i}} - a_{\textit{i-1}} &\text{$i \neq 1$}
 \end{cases}
\end{equation*}
Так как $a_i>a_{i-1}$, $b_i>0,~ b_i \in N$. Разность двух натуральных чисел определена единственным образом, т.е. $f$ - функция. Кроме того, $\forall\{b_n\}$ принадлежат множеству последовательностей нат. чисел.\newline
Тогда 
 \begin{equation*}
f^{-1} : 
 \begin{cases}
   a_{\textit{1}} = b_{\textit{1}} &\text{i = 1}\\
   a_{\textit{i}} = \sum\nolimits_{j=1}^i b_j &\text{$i \neq 1$}
 \end{cases}
\end{equation*}
 Достаточно легко заметить, что данное выражение функционально. $\forall\{a_n\}$ принадлежат множеству возрастающих последовательностей нат. чисел.
 Покажем, что $f \circ f^{-1}=Id:$
 
 \begin{equation*}
f \circ f^{-1} : 
 \begin{cases}
   c_{\textit{1}} = b_{\textit{1}} = a_{\textit{1}} &\text{i = 1}\\
   c_{\textit{i}} = a_1+ \sum\nolimits_{j=2}^i (a_j-a_{j-1}) = a_i &\text{$i \neq 1$}
 \end{cases}
 =Id
\end{equation*}
 Таким образом, получилось построить явную биекцию между конечными строго возрастающими последовательностями натуральных чисел и конечными последовательностями натуральных чисел.
\end{document}







